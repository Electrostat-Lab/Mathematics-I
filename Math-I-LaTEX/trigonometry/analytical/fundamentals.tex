This section is devoted to the deduction of the fundamental trigonometric identities 
from the unit circle, and the Pythagorean theorem.

First off, here is an exhaustive list of the fundamental trigonometric identities before heading 
to their deduction, although not commonly encountered, but it's useful in particular 
scenarios in software design:

If angle \( \angle{\theta} \) is a central angle inscribed by a unit circle \( C \),
with its initial side coincident with the positive direction of the x-axis, and the 
terminal side lies in any of the 4 quadrants in an \( R^2 \) vector-space, then the following holds:

\begin{itemize}
    \item \( \tan{\theta} = \frac{\sin{\theta}}{\cos{\theta}} \)
    \item \( \cot{\theta} = \frac{1}{\tan{\theta}} = \frac{\cos{\theta}}{\sin{\theta}} \)
    \item \( \csc{\theta} = \frac{1}{\sin{\theta}} \)
    \item \( \sec{\theta} = \frac{1}{\cos{\theta}} \)
    \item \( \sin^2{\theta} + \cos^2{\theta} = r^2 = 1 \)
    \item \( \tan^2{\theta} + 1 = \sec^2{\theta} \)
    \item \( 1 + \cot^2{\theta} = \csc^2{\theta} \)
\end{itemize}

\specialsubsection{Alternative forms for the fundamental identities:}{
\begin{itemize}
    \item \( \tan{\theta} = \frac{\sec{\theta}}{\csc{\theta}} \)
    \item \( \sin^2{\theta} = 1 - \cos^2{\theta} \)
    \item \( \cos^2{\theta} = 1 - \sin^2{\theta} \)
    \item \( \tan^2{\theta} = \sec^2{\theta} - 1 \)
    \item \( \cot^2{\theta} = \csc^2{\theta} - 1 \)
\end{itemize}
}{black}