This section concludes proofs and formulas for the Sine and Cosine interaction as Co-functions.
\\

\specialsubsection{Co-function Formulas}{
The following is the summary of the re-usable formulas, notice how these will be utilize dramatically
in the subsequent sections, if \(\angle{\theta}\) is an acute angle, then the following holds true:
\\
\begin{equation}
    \cos{\theta} = \sin{(\frac{\pi}{2} - \theta)}
    \label{eq:equation}
\end{equation}

\begin{equation}
    \sin{\theta} = \cos{(\frac{\pi}{2} - \theta)}
    \label{eq:equation2}
\end{equation}

\begin{equation}
    \tan{\theta} = \frac{\sin{\theta}}{\cos{\theta}} \\
                 = \frac{\cos{(\frac{\pi}{2} - \theta})}{\sin{(\frac{\pi}{2} - \theta})} \\
    \label{eq:equation3}
\end{equation}

\begin{equation}
    \sin{(\theta + \frac{\pi}{2})} = \cos{(\frac{\pi}{2} - (\theta + \frac{\pi}{2}))} \\
                                 = \cos{(-\theta)} = \cos{\theta}
    \label{eq:equation4}
\end{equation}

\(\because \text{\(\angle{(\theta + \frac{\pi}{2})}\) and \(\angle{\alpha}\) are supplementary angles (i.e., \(\angle{(\theta + \frac{\pi}{2})} +  \angle{\alpha} = \angle{\pi}).\)}\\
    \therefore \sin{(\theta + \frac{\pi}{2})} = \sin{(\pi - \alpha)} \\
                                              = \sin{(\pi - \alpha)} \\
\)

\begin{equation}
    \therefore \sin{(\theta + \frac{\pi}{2})} = \sin{(\pi - \alpha)}
    \label{eq:equation5}
\end{equation}

\begin{equation}
    \sin{(\theta + \pi)} = \sin{(3\frac{\pi}{2} - \alpha)}
    \label{eq:equation6}
\end{equation}

}{red}

\specialsubsection{Proof for the co-functions}{

}{black}